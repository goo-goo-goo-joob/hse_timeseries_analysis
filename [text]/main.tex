%!TEX TS-program = xelatex

\documentclass[a4paper,12pt]{article}

\input{data/preambular.tex}
\begin{document}
\begin{titlepage}
	\begin{center}
		ПРАВИТЕЛЬСТВО РОССИЙСКОЙ ФЕДЕРАЦИИ \\
 		ФЕДЕРАЛЬНОЕ  ГОСУДАРСТВЕННОЕ АВТОНОМНОЕ \\
		ОБРАЗОВАТЕЛЬНОЕ УЧРЕЖДЕНИЕ ВЫСШЕГО ОБРАЗОВАНИЯ\\
		«НАЦИОНАЛЬНЫЙ ИССЛЕДОВАТЕЛЬСКИЙ УНИВЕРСИТЕТ\\
		«ВЫСШАЯ ШКОЛА ЭКОНОМИКИ»
	\end{center}
	
	\begin{center}
		\textbf{Московский институт электроники и математики}
		
		\textbf{Им. А.Н.Тихонова НИУ ВШЭ}
		
		\vspace{2ex}
		
		\textbf{Направление 01.03.04. Прикладная математика \\
			Бакалаврская программа <<Прикладная математика>>}
	\end{center}
	\vspace{1ex}	
	
	\vspace{1ex}
	\begin{center}
		\textbf{Отчет по самостоятельной работе \\
			по дисциплине <<Методы анализа стохастических взаимосвязей>>\\
			часть 2
	}
	\end{center}	

	\vspace{2ex}
	\vfill
	
	\vspace{2ex}
	
	\begin{flushright}
		\textbf{Бригада №7:}
		
		\vspace{2ex}
		
		Ремизова Анна Петровна, 4 курс, БПМ174
		
		Самоделкина Мария Владимировна, 4 курс, БПМ174

	\end{flushright}

	\vspace{5ex}
	\begin{center}
		Москва \the\year \, г.
	\end{center}
	
\end{titlepage}
\addtocounter{page}{1}
\tableofcontents
\pagebreak

\section{Описание данных}
% Период времени и тип данных
Частота измерений каждые 4 секунды. Величину стандартного интервала наблюдений принимаем равным среднему времени прохождения спортсменом одного круга (около 6 минут).

% Гипотезы

\subsection{Зависимая переменная Пульс}
% Определение
\begin{figure}[H]
	\centering
	\includegraphics[width=0.7\linewidth]{../[graphics]/hr_graph}
	\caption{График зависимой переменной Пульс}
	\label{fig:hr_graph}
\end{figure}

% Анализ автокорреляции
\textbf{\textit{Анализ автокорреляции}}

\begin{figure}[H]
	\centering
	\includegraphics[width=0.7\linewidth]{../[graphics]/hr_acf_100}
	\caption{Графики ACF и PACF зависимой переменной Пульс}
	\label{fig:hr_acf_100}
\end{figure}

График автокорреляции (рис. \ref{fig:hr_acf_100}) убывает медленно, предполагается наличие тренда. На графике видны колебания с периодом в треть круга (лаг 30), полкруга (лаг 45) и круг (лаг 88). При анализе графика значений зависимой переменной (рис. \ref{fig:hr_graph}) также хорошо видны круговые колебания. По виду графиков можно говорить о наличии периодического тренда с периодом, равным среднему времени прохождения спортсменом круга.

% Анализ спектрограммы
\textbf{\textit{Анализ спектрограммы}}

\begin{figure}[H]
	\centering
	\includegraphics[width=0.7\linewidth]{../[graphics]/hr_spectr}
	\caption{График периодограммы зависимой переменной Пульс}
	\label{fig:hr_spectr}
\end{figure}

При анализе графика (рис. \ref{fig:hr_spectr}) видны пики вблизи нулевой частоты, они свидетельствуют о потенциальном наличии долгосрочного тренда. На графике также видны несколько пиков в циклической области спектра, следовательно, в данном ряде могут присутствовать циклы. 
В сезонной части спектра на частоте $f = 0,07121 \approx \frac{\pi}{44}$ значение спектральной частоты равно $Sd = 2847,9$, указанная частота соответствует периоду круга. 
На частоте $f = 0,14242 \approx \frac{\pi}{22}$ значение спектральной частоты равно $Sd = 665,40$, указанная частота соответствует периоду полукруга. 
На частоте $f = 0,21363 \approx \frac{\pi}{15}$ значение спектральной частоты равно $Sd = 730,46$, указанная частота соответствует периоду в треть круга. Гармоническая составляющая в указанном ряду является хорошо выраженной.

% Анализ стационарности с использованием критерия Dickey-Fuller

%Выводы
\textbf{\textit{Выводы}}

Наличие долгосрочного тренда зависимой переменной Пульса объясняется тем, что с течением времени тренировки спортсмен устает, вследствие чего пульс постепенно увеличивается. Наличие кругового тренда объясняется тем, что рельеф местности каждый круг повторяется. Наличие сезонной частоты с периодом, равным половине круга связано с особенностями местности: каждые полкруга высота траектории движения сначала возрастала, а затем убывала. Наличие сезонной частоты с периодом, равным трети круга связано с периодичной сменой техники передвижения. %TODO

\subsection{Независимые переменные}
\subsubsection{Высота}
% Определение
\begin{figure}[H]
	\centering
	\includegraphics[width=0.7\linewidth]{../[graphics]/ele_graph}
	\caption{График зависимой Высота}
	\label{fig:ele_graph}
\end{figure}

% Анализ автокорреляции
\textbf{\textit{Анализ автокорреляции}}

\begin{figure}[H]
	\centering
	\includegraphics[width=0.7\linewidth]{../[graphics]/ele_acf_100}
	\caption{Графики ACF и PACF переменной Высота}
	\label{fig:ele_acf_100}
\end{figure}

График автокорреляции (рис. \ref{fig:ele_acf_100}) убывает медленно, предполагается наличие тренда. На графике видны колебания с периодом полкруга (лаг 45) и круг (лаг 88). При анализе графика значений переменной (рис. \ref{fig:ele_graph}) также хорошо видны круговые колебания. По виду графиков можно говорить о наличии периодического тренда с периодом, равным среднему времени прохождения спортсменом круга.

% Анализ спектрограммы
\textbf{\textit{Анализ спектрограммы}}

\begin{figure}[H]
	\centering
	\includegraphics[width=0.7\linewidth]{../[graphics]/ele_spectr}
	\caption{График периодограммы переменной Высота}
	\label{fig:ele_spectr}
\end{figure}

При анализе графика (рис. \ref{fig:hr_spectr}) видно, что пики вблизи нулевой частоты отсутствуют, долгосрочного тренда нет. На графике отсутствуют пики в циклической области спектра. 
В сезонной части спектра на частоте $f = 0,07121 \approx \frac{\pi}{44}$ значение спектральной частоты равно $Sd = 1021,2$, указанная частота соответствует периоду круга. У данного ряда есть ярковыраженная сезонная составляющая.
Также имеются другие неярковыраженные пики в сезонной части спектра.

% Анализ стационарности с использованием критерия Dickey-Fuller

%Выводы
\textbf{\textit{Выводы}}

Долгосрочного тренда в рассматриваемом ряде быть не должно: спортсмен каждый круг проезжал по одной и той же территории, высота не изменялась. 
Однако тренд в течение круга есть. Видно, что за круг высота сначала растет, а потом практически симметрично падает. Такая симметрия связана с особенностями местности, на которой снимались показания.
Наличие кругового тренда объясняется тем, что рельеф местности каждый круг повторяется. Однако в изменениях высоты могут присутствовать незначительные колебания из-за того, что спортсмен каждый раз мог выбирать незначительно отличающуюся траекторию.

\subsubsection{Каденс}
% Определение
\begin{figure}[H]
	\centering
	\includegraphics[width=0.7\linewidth]{../[graphics]/cad_graph}
	\caption{График зависимой Каденс}
	\label{fig:cad_graph}
\end{figure}

% Анализ автокорреляции
\textbf{\textit{Анализ автокорреляции}}

\begin{figure}[H]
	\centering
	\includegraphics[width=0.7\linewidth]{../[graphics]/cad_acf_100}
	\caption{Графики ACF и PACF переменной Каденс}
	\label{fig:cad_acf_100}
\end{figure}

График автокорреляции (рис. \ref{fig:cad_acf_100}) убывает быстро, предполагается отсутствие тренда. Влияние сезонности предположительно отсутствует.

% Анализ спектрограммы
\textbf{\textit{Анализ спектрограммы}}

\begin{figure}[H]
	\centering
	\includegraphics[width=0.7\linewidth]{../[graphics]/cad_spectr}
	\caption{График периодограммы переменной Каденс}
	\label{fig:cad_spectr}
\end{figure}

При анализе графика (рис. \ref{fig:cad_spectr}) видно, что пики вблизи нулевой частоты отсутствуют, долгосрочного тренда нет. На графике присутствуют пики в циклической области спектра. 
В сезонной части спектра на частоте $f = 0,21782 \approx \frac{\pi}{14}$ значение спектральной частоты равно $Sd = 1331,6$, указанная частота соответствует периоду трети круга.
Также имеются другие пики в сезонной части спектра.

% Анализ стационарности с использованием критерия Dickey-Fuller

%Выводы
\textbf{\textit{Выводы}}

Отсутствие тренда связано с тем, что используемая в конкретный момент техника передвижения спортсмена практически не зависит от техники, используемой в предыдущие моменты времени.
Наличие сезонной частоты с периодом, равным трети круга связано с периодичностью применяемой техники передвижения. %TODO

\subsubsection{Сезонные переменные}

\end{document} % конец документа